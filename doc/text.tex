% \part{Erster Teil} % Teile nur bei sehr langen Arbeiten

\chapter{Einleitung}

Zweck dieser fast inhaltsleeren Zeilen ist es, den Aufbau und das
typographische Erscheinungsbild einer studentischen Arbeit unseres
Instituts vorzustellen. Für \LaTeX-Benutzer gibt eine Klasse.
Autoren, die \glqq{}Word\grqq{} oder andere Textverarbeitungssysteme
verwenden, können dieses Dokument \zB dazu benutzen, Seitenränder,
Abstände und Schriftgrößen auszumessen.\nocite*{prog}

Eine (studentische) Arbeit besteht aus Titelei, Textteil und Anhang.

\begin{center}\small
\begin{tabular}{|l|l|}\hline
Titel & eigenes Titelblatt\\
Eidesstattliche Erklärung & für Abschlussarbeiten\\
Zusammenfassung & \glqq{}Aushängeschild\grqq, max. 1 Seite\\
Vorwort & optional, max. 1 Seite\\
Inhaltsverzeichnis & obligatorisch\\
Verzeichnis der Tabellen & optional, ab hier im Inhaltsverzeichnis\\
Verzeichnis der Abbildungen & optional\\\hline
1. Einleitung, Problemstellung, Motivation & Einbindung in das wiss. Umfeld\\
2. Grundlagen, Theorie, Vorarbeiten & benötigte Voraussetzungen\\
3. Eigene Arbeiten & \glqq{}Lösungsstrategie\grqq\\
4. Ergebnisse & \glqq{}Neuigkeiten\grqq\\
5. Bewertung, Ausblick & Schlussfolgerungen und \glqq{}Botschaft\grqq\\\hline
Literatur & obligatorisch\\
Glossar & optional\\
Sachwortregister & optional\\\hline
\end{tabular}
\end{center}

\noindent
Bücher enthalten häufig vor der eigentlichen Titelseite ein Deckblatt,
den sogenannten Schmutztitel. Er enthält nur eine kurze Angabe von Titel
und Verfasser. Das Vorwort\footnote{In \cite{lit01} heißt es: \glqq{}Ein
Vorwort beschränkt sich bei studentischen Arbeiten im Wesentlichen auf
die Danksagung an Personen, die z.\,B. durch persönliche Begleitung,
Dienstleistungen oder Anregungen die Arbeit unterstützt haben. Ein Vorwort
ist keine Inhaltsangabe oder Kurzfassung. Dem Betreuer, dem die Arbeit ja
auch zugute kommt, muss nicht gedankt werden.\grqq} ist optional und kann
auch nach den Verzeichnissen stehen. Weitere Verzeichnisse, z.\,B. für
Abkürzungen, Definitionen oder Sätze, sind denkbar. Im Allgemeinen
erfolgt die Paginierung in der Titelei mit römischen und ab
dem Textteil mit arabischen Ziffern. Der Textteil variiert naturgemäß
stark. Der angebene Aufbau ist daher nur als Vorschlag zu sehen. Der
Anhang kann weitere Teile, z.\,B. Programm-Listen, enthalten.
\emph{Der Text auf den folgenden Seiten dient nur der Illustration
des Satzspiegels, der Schriftgrößen, der Gestaltung der Fußnoten u.\,Ä.}

% \part{Zweiter Teil} % Teile nur bei sehr langen Arbeiten

\chapter{Theoretische Grundlagen}\label{Kapitel}

\section{Ein Abschnitt}\label{Sektion}

\paragraph{Semantik}\label{Paragraf}

Anfang der 1980er Jahre entwickelten E.\,G. Manes und M.\,A. Arbib die
additive Semantik zur Definition von Programmiersprachen. Es handelt
sich hierbei um eine denotationale Vorgehensweise, die auf additiven
Monoiden basiert und eine Behandlung im Rahmen der
Kategorientheorie zulässt. Hier geht's weiter mit \cite{lit01}.

\subsection{Ein Unterabschnitt}

Algebraische Theorien einer Kategorie bestehen aus einem Funktor und
aus natürlichen Transformationen, die bestimmte Gesetze erfüllen. In
diesem Vortrag werden spezielle algebraische Theorien definiert, die
Matrixtheorien mit Prädikat, und es wird gezeigt, dass diese Theorien
eine große Klasse von Programmen im Rahmen der additiven Semantik
beschreiben können. Es kostet \EUR{5,45} mehr als letzte Woche.

\paragraph{Algebraische Theorien}

Algebraische Theorien $\mathcal{T}=(T,\eta,\mu)$ in einer Kategorie
$\mathcal{K}$ bestehen aus einem Funktor $T$ und aus natürlichen
Transformationen $\eta$ und $\mu$, die bestimmte Gesetze erfüllen. In
diesem Vortrag werden spezielle algebraische Theorien definiert, die
Matrixtheorien mit Prädikat, und es wird gezeigt, dass diese Theorien
eine große Klasse von Programmen im Rahmen der additiven Semantik
beschreiben können. Ein Test für das Glossar: Gruppe\gruppe.

Algebraische Theorien einer Kategorie bestehen aus einem Funktor und
aus natürlichen Transformationen, die bestimmte Gesetze erfüllen. In
diesem Vortrag werden spezielle algebraische Theorien definiert, die
Matrixtheorien mit Prädikat, und es wird gezeigt, dass diese Theorien
eine große Klasse von Programmen im Rahmen der additiven Semantik
beschreiben können. Ein Test für Index und Glossar:
Kategorientheorie,\index{Kategorientheorie}
algebraische Theorie\algtheorie

Anfang der 1980er Jahre entwickelten E.\,G. Manes und M.\,A. Arbib die
additive Semantik zur Definition von Programmiersprachen. Es handelt
sich hierbei um eine denotationale Vorgehensweise, die auf additiven
Monoiden basiert und eine Behandlung im Rahmen der Kategorientheorie
zulässt.

\paragraph{Ein weiterer Paragraf}

Algebraische Theorien einer Kategorie bestehen aus einem Funktor und
aus natürlichen Transformationen, die bestimmte Gesetze erfüllen. In
diesem Vortrag werden spezielle algebraische Theorien definiert, die
Matrixtheorien mit Prädikat, und es wird gezeigt, dass diese Theorien
eine große Klasse von Programmen im Rahmen der additiven Semantik
beschreiben können.

Algebraische Theorien einer Kategorie bestehen aus einem Funktor und
aus natürlichen Transformationen, die bestimmte Gesetze erfüllen. In
diesem Vortrag werden spezielle algebraische Theorien definiert, die
Matrixtheorien mit Prädikat, und es wird gezeigt, dass diese Theorien
eine große Klasse von Programmen im Rahmen der additiven Semantik
beschreiben können.

\section{Ein weiterer Abschnitt}

\paragraph{Semantik}

Anfang der 1980er Jahre entwickelten E.\,G. Manes und M.\,A. Arbib die
additive Semantik zur Definition von Programmiersprachen. Es handelt
sich hierbei um eine denotationale Vorgehensweise, die auf additiven
Monoiden basiert und eine Behandlung im Rahmen der Kategorientheorie
zulässt. Hier geht's weiter mit \cite{lit01}.\gruppe\abelsch

Er sagte: {\glqq}Sag' doch nicht immer: {\glq}Nein, danke!{\grq}!{\grqq}

Sie sagte: »Sag' doch nicht immer: {\frq}Ja, bitte!{\flq}!«

Algebraische Theorien einer Kategorie bestehen aus einem Funktor und
aus natürlichen Transformationen, die bestimmte Gesetze erfüllen. In
diesem Vortrag werden spezielle algebraische Theorien definiert, die
Matrixtheorien mit Prädikat, und es wird gezeigt, dass diese Theorien
eine große Klasse von Programmen im Rahmen der additiven Semantik
beschreiben können. Ein Test für den Index:
Kategorientheorie\index{Kategorientheorie} und
das Literaturverzeichnis \cite{lit02,lit03}.

Algebraische Theorien $\mathcal{T}=(T,\eta,\mu)$ in einer Kategorie
$\mathcal{K}$ bestehen aus einem Funktor $T$ und aus natürlichen
Transformationen $\eta$ und $\mu$, die bestimmte Gesetze erfüllen. In
diesem Vortrag werden spezielle algebraische Theorien definiert, die
Matrixtheorien mit Prädikat, und es wird gezeigt, dass diese Theorien
eine große Klasse von Programmen im Rahmen der additiven Semantik
beschreiben können.

\paragraph{Algebraische Theorien}

Algebraische Theorien $\mathcal{T}=(T,\eta,\mu)$ in einer Kategorie
$\mathcal{K}$ bestehen aus einem Funktor $T$ und aus natürlichen
Transformationen $\eta$ und $\mu$, die bestimmte Gesetze erfüllen. In
diesem Vortrag werden spezielle algebraische Theorien definiert, die
Matrixtheorien mit Prädikat, und es wird gezeigt, dass diese Theorien
eine große Klasse von Programmen im Rahmen der additiven Semantik
beschreiben können.\algtheorie

Algebraische Theorien einer Kategorie bestehen aus einem Funktor und
aus natürlichen Transformationen, die bestimmte Gesetze erfüllen. In
diesem Vortrag werden spezielle algebraische Theorien definiert, die
Matrixtheorien mit Prädikat, und es wird gezeigt, dass diese Theorien
eine große Klasse von Programmen im Rahmen der additiven Semantik
beschreiben können.

Anfang der 1980er Jahre entwickelten E.\,G. Manes und M.\,A. Arbib die
additive Semantik zur Definition von Programmiersprachen. Es handelt
sich hierbei um eine denotationale Vorgehensweise, die auf additiven
Monoiden basiert und eine Behandlung im Rahmen der Kategorientheorie
zulässt.

Algebraische Theorien einer Kategorie bestehen aus einem Funktor und
aus natürlichen Transformationen, die bestimmte Gesetze erfüllen. In
diesem Vortrag werden spezielle algebraische Theorien definiert, die
Matrixtheorien mit Prädikat, und es wird gezeigt, dass diese Theorien
eine große Klasse von Programmen im Rahmen der additiven Semantik
beschreiben können.

\paragraph{Aufgabe}

Algebraische Theorien einer Kategorie bestehen aus einem Funktor und
aus natürlichen Transformationen, die bestimmte Gesetze erfüllen. In
diesem Vortrag werden spezielle algebraische Theorien definiert, die
Matrixtheorien mit Prädikat, und es wird gezeigt, dass diese Theorien
eine große Klasse von Programmen im Rahmen der additiven Semantik
beschreiben können.

Gegeben seien die Ebenen $E_1$ und $E_2$
im $\mathbb{R}^3$. Man bestimme ihre Schnittmenge $E_1\cap E_2$:
$$E_1:\mathfrak{x}=
\Vector{1}{2}{3}+\lambda\Vector{4}{5}{6}+\mu\Vector{7}{8}{9}$$
$$E_2:\mathfrak{x}=
\Vector{0}{2}{0}+\lambda\Vector{1}{-5}{0}+\mu\Vector{1}{8}{-9}$$
Die korrekte Schreibweise f"ur die Ebene $E_1$ lautet:
$$E_1=\left\{\quad\mathfrak{x}\in\mathbb{R}^3\quad\left|\quad\mathfrak{x}=
\Vector{1}{2}{3}+\lambda\Vector{4}{5}{6}+\mu\Vector{7}{8}{9},\quad
\lambda,\mu\in\mathbb{R}\quad\right.\right\}$$

\paragraph{Der nächste Paragraf}

Anfang der 1980er Jahre entwickelten E.\,G. Manes und M.\,A. Arbib die
additive Semantik zur Definition von Programmiersprachen. Es handelt
sich hierbei um eine denotationale Vorgehensweise, die auf additiven
Monoiden basiert und eine Behandlung im Rahmen der Kategorientheorie
zulässt. Hier geht's weiter.\algtheorie

$$\int \cos(x)\,dx = \sin(x)+c$$

Algebraische Theorien einer Kategorie bestehen aus einem Funktor und
aus natürlichen Transformationen, die bestimmte Gesetze erfüllen. In
diesem Vortrag werden spezielle algebraische Theorien definiert, die
Matrixtheorien mit Prädikat, und es wird gezeigt, dass diese Theorien
eine große Klasse von Programmen im Rahmen der additiven Semantik
beschreiben können.

Anfang der 1980er Jahre entwickelten E.\,G. Manes und M.\,A. Arbib die
additive Semantik zur Definition von Programmiersprachen. Es handelt
sich hierbei um eine denotationale Vorgehensweise, die auf additiven
Monoiden basiert und eine Behandlung im Rahmen der Kategorientheorie
zulässt. Hier geht's weiter.

Algebraische Theorien einer Kategorie bestehen aus einem Funktor und
aus natürlichen Transformationen, die bestimmte Gesetze erfüllen. In
diesem Vortrag werden spezielle algebraische Theorien definiert, die
Matrixtheorien mit Prädikat, und es wird gezeigt, dass diese Theorien
eine große Klasse von Programmen im Rahmen der additiven Semantik
beschreiben können.

\section{Noch ein Abschnitt}

Anfang der 1980er Jahre entwickelten E.\,G. Manes und M.\,A. Arbib die
additive Semantik zur Definition von Programmiersprachen. Es handelt
sich hierbei um eine denotationale Vorgehensweise, die auf additiven
Monoiden basiert und eine Behandlung im Rahmen der Kategorientheorie
zulässt. Hier geht's weiter mit \cite{lit01}. Die Semantik wird auf Seite
Semantik erklärt.

Vergleiche \ref{Kapitel}--\ref{Sektion}--\ref{Paragraf} und
\pageref{Kapitel}--\pageref{Sektion}--\pageref{Paragraf}.

Algebraische Theorien einer Kategorie bestehen aus einem Funktor und
aus natürlichen Transformationen, die bestimmte Gesetze erfüllen. In
diesem Vortrag werden spezielle algebraische Theorien definiert, die
Matrixtheorien mit Prädikat, und es wird gezeigt, dass diese Theorien
eine große Klasse von Programmen im Rahmen der additiven Semantik
beschreiben können.

\begin{multicols}{2}
\paragraph{Ein zweispaltiger Absatz}
Algebraische Theorien einer Kategorie bestehen aus einem Funktor und
aus natürlichen Transformationen, die bestimmte Gesetze erfüllen. In
diesem Vortrag werden spezielle algebraische Theorien definiert, die
Matrixtheorien mit Prädikat, und es wird gezeigt, dass diese Theorien
eine große Klasse von Programmen im Rahmen der additiven Semantik
beschreiben können.
Algebraische Theorien einer Kategorie bestehen aus einem Funktor und
aus natürlichen Transformationen, die bestimmte Gesetze erfüllen. In
diesem Vortrag werden spezielle algebraische Theorien definiert, die
Matrixtheorien mit Prädikat, und es wird gezeigt, dass diese Theorien
eine große Klasse von Programmen im Rahmen der additiven Semantik
beschreiben können.
\end{multicols}

\begin{figure}[htpb]
\begin{center}
\setlength{\unitlength}{10mm}
\begin{picture}(3,4)
\put(0.0,0.0){\line(1,0){3}}
\put(3.0,0.0){\line(0,1){4}}
\put(0.0,0.0){\line(3,4){3}}
\end{picture}
\end{center}
\caption{Dreieck}
\end{figure}

Algebraische Theorien einer Kategorie bestehen aus einem Funktor und
aus natürlichen Transformationen, die bestimmte Gesetze erfüllen. In
diesem Vortrag werden spezielle algebraische Theorien definiert, die
Matrixtheorien mit Prädikat, und es wird gezeigt, dass diese Theorien
eine große Klasse von Programmen im Rahmen der additiven Semantik
beschreiben können.\algtheorie

\begin{figure}[htpb]
\begin{center}
\setlength{\unitlength}{10mm}
\begin{picture}(1,1)
\put(0.0,0.0){\line(1,0){1}}
\put(0.0,0.0){\line(0,1){1}}
\put(1.0,0.0){\line(0,1){1}}
\put(0.0,1.0){\line(1,0){1}}
\end{picture}
\caption{Kleines Quadrat}
\end{center}
\end{figure}

Algebraische Theorien einer Kategorie bestehen aus einem Funktor und
aus natürlichen Transformationen, die bestimmte Gesetze erfüllen. In
diesem Vortrag werden spezielle algebraische Theorien definiert, die
Matrixtheorien mit Prädikat, und es wird gezeigt, dass diese Theorien
eine große Klasse von Programmen im Rahmen der additiven Semantik
beschreiben können.

\begin{table}[htpb]
\begin{center}
\begin{tabular}{|l|r|}\hline
\texttt{alpha} & $\alpha$ \\\hline
\texttt{beta}  & $\beta$  \\\hline
\texttt{gamma} & $\gamma$ \\\hline
\end{tabular}
\end{center}
\caption{Alpha Beta Gamma}
\end{table}

Algebraische Theorien einer Kategorie bestehen aus einem Funktor und
aus natürlichen Transformationen, die bestimmte Gesetze erfüllen. In
diesem Vortrag werden spezielle algebraische Theorien definiert, die
Matrixtheorien mit Prädikat, und es wird gezeigt, dass diese Theorien
eine große Klasse von Programmen im Rahmen der additiven Semantik
beschreiben können. Algebraische Theorien einer Kategorie bestehen
aus einem Funktor und aus natürlichen Transformationen, die bestimmte
Gesetze erfüllen.

\begin{table}[htpb]
\begin{center}
\begin{tabular}{|l|r|}\hline
\texttt{delta} & $\delta$ \\\hline
\texttt{mu}    & $\mu$    \\\hline
\texttt{sigma} & $\sigma$ \\\hline
\end{tabular}
\caption{Delta Mu Sigma}
\end{center}
\end{table}

Algebraische Theorien einer Kategorie bestehen aus einem Funktor und
aus natürlichen Transformationen, die bestimmte Gesetze erfüllen. In
diesem Vortrag werden spezielle algebraische Theorien definiert, die
Matrixtheorien mit Prädikat, und es wird gezeigt, dass diese Theorien
eine große Klasse von Programmen im Rahmen der additiven Semantik
beschreiben können.\algtheorie\begriff Algebraische Theorien einer
Kategorie bestehen aus einem Funktor und aus natürlichen Transformationen,
die bestimmte Gesetze erfüllen.\algebra\topologie\geometrie
\zahlentheorie\stochastik\analysis\software\algorithmen
\netze\theorie\linear\logik\programmieren
\aumlaut\oumlaut\uumlaut

\paragraph{Test des Sachwortregisters} In diesem Paragrafen testen
wir das Sachwortregister durch einige Einträge:

Addition\index{Addition}
Baeq\index{Baeq}
Baes\index{Baes}
Bär\index{Baer@Bär}
Injektion\index{Injektion|see{Addition}}
Klein\index{Klein}
klein\index{klein}
Aachen\index{Aachen}
zahlen\index{zahlen}
z.\,B.\index{z. B.@z.\,B.}

Haus\index{Haus}
Maus\index{Maus}
Hans\index{Hans}
Klaus\index{Klaus}
Saus\index{Saus}
Laus\index{Laus}
Caus\index{Caus}
Daus\index{Daus}
raus\index{raus}

Zahlen
7\index{7}
IV\index{4@IV}
X\index{10@X}
MCCL\index{1250@MCCL (1250)}
1370\index{1370}
8b\index{8@8b}
8a\index{8@8a}

Sonderzeichen
?\index{?}
!\index{+!}
Ausruf\index{Ausruf}
Ausrufezeichen\index{Ausrufezeichen}
@\index{+@}
$\otimes$\index{$\otimes$}

großes übles Ärgernis
\index{Aergernis@Ärgernis}
\index{Aergernis@Ärgernis!grosses@großes}
\index{Aergernis@Ärgernis!uebles@übles}
\index{Aergernis@Ärgernis!grosses@großes!uebles@übles}
\index{grosses Aergernis@großes Ärgernis}
\index{uebles Aergernis@übles Ärgernis}
\index{grosses uebles Aergernis@großes übles Ärgernis}

Æsop\index{Aesop@Æsop}
Masse\index{Masse}
Maße\index{Masse1@Maße}
Schroeder\index{Schroeder}
Schröder\index{Schroeder1@Schröder}
§ 51\index{§ 51}
§§ 51--54\index{§§ 51--54}

Algebraische Theorien einer Kategorie bestehen aus einem Funktor und
aus natürlichen Transformationen, die bestimmte Gesetze erfüllen. In
diesem Vortrag werden spezielle algebraische Theorien definiert, die
Matrixtheorien mit Prädikat, und es wird gezeigt, dass diese Theorien
eine große Klasse von Programmen im Rahmen der additiven Semantik
beschreiben können.\algtheorie\begriff Algebraische Theorien einer
Kategorie bestehen aus einem Funktor und aus natürlichen Transformationen,
die bestimmte Gesetze erfüllen.

\endinput
