%%% Einträge im Glossar werden durch den Befehl
%%%    \glossar{begriff}{erklärung}
%%% vorgenommen.

%Determinismus
%Lookahead
%Any-Übergang
%Else-Übergang
%spontaner Übergang
%Block
%boundingbox
%fsm, nea, dea
%hcz

\newcommand{\algtheorie}{\glossar{Algebraische Theorie}%
{Der Begriff \glqq{}algebraische Theorie\grqq{} entstammt der Kategorientheorie
und verallgemeinert dies und das.}}

\newcommand{\gruppe}{\glossar{Gruppe}{Eine \emph{Gruppe} ist eine
Menge mit einer zweistelligen assoziativen Verknüpfung mit Einselement
und inversen Elementen.}}

\newcommand{\abelsch}{\glossar{Abelsche Gruppe}{Eine Gruppe
heißt \emph{abelsch}, wenn die Verknüpfung kommutativ ist.}}

\newcommand{\begriff}{\glossar{Begriff}{und Erklärung.}}

\newcommand{\algebra}{\glossar{Algebra}{ist eine mathematische Disziplin.}}

\newcommand{\topologie}{\glossar{Topologie}{ist eine mathematische Disziplin.}}

\newcommand{\zahlentheorie}{\glossar{Zahlentheorie}{ist eine mathematische Disziplin.}}

\newcommand{\geometrie}{\glossar{Geometrie}{ist eine mathematische Disziplin.}}

\newcommand{\stochastik}{\glossar{Stochastik}{ist eine mathematische Disziplin.}}

\newcommand{\analysis}{\glossar{Analysis}{ist eine mathematische Disziplin.}}

\newcommand{\algorithmen}{\glossar{Algorithmen}{und Datenstrukturen
ist ein Teilgebiet der Informatik.}}

\newcommand{\theorie}{\glossar{Theoretische Informatik}
{ist ein Teilgebiet der Informatik, das alle Studierende hören müssen.}}

\newcommand{\netze}{\glossar{Betriebssysteme und Netze}
{ist ein Teilgebiet der Informatik.}}

\newcommand{\software}{\glossar{Software-Technik}
{ist ein Teilgebiet der Informatik, das alle Studierende hören müssen.}}

\newcommand{\linear}{\glossar{Lineare Algebra}
{ist eine Vorlesung, die alle Studierende hören müssen.}}

\newcommand{\logik}{\glossar{Logik}
{ist eine Vorlesung, die alle Studierende hören müssen.}}

\newcommand{\programmieren}{\glossar{Programmieren}
{ist eine Vorlesung, die alle Studierende hören müssen.}}

\newcommand{\aumlaut}{\glossar{Ä}{ein Eintrag, der mit einem Umlaut beginnt.}}

\newcommand{\oumlaut}{\glossar{Ö}{ein Eintrag, der mit einem Umlaut beginnt.}}

\newcommand{\uumlaut}{\glossar{Ü}{ein Eintrag, der mit einem Umlaut beginnt.}}
