Das Ziel dieser Arbeit war es ein didaktisches Werkzeug zu entwickeln, das in erster Linie Studenten hilft, die Funktionsweise von endlichen Automaten zu verstehen und spielerisch zu erlernen. Wert wird also auf eine gute visuelle Darstellung und ein komfortabel zu bedienendes Interface gelegt. Es geht nicht darum, ein Programm zu entwickeln, das es ermöglicht, Automaten schnell und effektiv zu simulieren.

Wichtige Aspekte sind dabei ist eine graphische und tabellarische Darstellung der Automaten, eine komfortable Möglichkeit Automaten im Programm zu konstruieren und dann Eingaben für den Automaten schrittweise zu simulieren. Auch eine Simulation der Konstruktion, also ein schrittweiser Aufbau eines vorgefertigten Automaten gehört dazu.

Der theoretische Teil der Arbeit beschäftigt sich mit universellen Abstandsautomaten. Diese sollen in das entwickelte Programm integriert werden und mit deren Funktionen konstruiert, simuliert und erkundet werden. Diese Abstandsautomaten stellen besondere Erfordernisse an die Simulation. Benötigt werden Aspekte wie Nichtdeterminismus, spontane Übergänge, „Joker-Übergänge“, die für jeden Buchstaben benutzt werden können, und die Möglichkeit Übergangslabels einer Länge größer als eins zuzulassen.

Weiterhin sind einige Algorithmen für Automaten implementiert wie Epsilon-Eli\-mination und Potenzmengenkonstruktion.
