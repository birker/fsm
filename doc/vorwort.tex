Ein Vorwort ist in einer studentischen Arbeit nicht unbedingt
erforderlich.  Im Dudentaschenbuch \cite{lit01} heißt es auf
Seite~30 wörtlich:

\begin{quote}\small
Ein Vorwort beschränkt sich bei studentischen Arbeiten
im Wesentlichen auf die Danksagung an Personen, die
z.\,B. durch persönliche Begleitung, Dienstleistungen
oder Anregungen die Arbeit unterstützt haben. Ein
Vorwort ist keine Inhaltsangabe oder Kurzfassung.
Es kann vor oder nach dem Inhaltsverzeichnis
erscheinen, ist jedoch nicht unbedingt nötig und
bei verschiedenen Instituten auch nicht üblich.
Dem Betreuer, dem die Arbeit ja auch zugute kommt,
muss nicht gedankt werden. Das Vorwort darf maximal
eine Seite umfassen.
\end{quote}

Ein Test für Literaturverzeichnis, Index und Glossar:
(abelsche) Gruppe, Kategorientheorie
\cite{lit04,lit01,lit02,lit03}%
\gruppe\abelsch\index{Kategorientheorie}.
Hier steht eine URL, die man in der DVI- oder PDF-Datei
auch anklicken kann:
\href{http://www.cs.tu-bs.de/}{WWW-Seiten der Informatik}
