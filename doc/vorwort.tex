Ein Vorwort ist in einer studentischen Arbeit nicht unbedingt
erforderlich.  Im Dudentaschenbuch \cite{lit01} hei�t es auf
Seite~30 w�rtlich:

\begin{quote}\small
Ein Vorwort beschr�nkt sich bei studentischen Arbeiten
im Wesentlichen auf die Danksagung an Personen, die
z.\,B. durch pers�nliche Begleitung, Dienstleistungen
oder Anregungen die Arbeit unterst�tzt haben. Ein
Vorwort ist keine Inhaltsangabe oder Kurzfassung.
Es kann vor oder nach dem Inhaltsverzeichnis
erscheinen, ist jedoch nicht unbedingt n�tig und
bei verschiedenen Instituten auch nicht �blich.
Dem Betreuer, dem die Arbeit ja auch zugute kommt,
muss nicht gedankt werden. Das Vorwort darf maximal
eine Seite umfassen.
\end{quote}

Ein Test f�r Literaturverzeichnis, Index und Glossar:
(abelsche) Gruppe, Kategorientheorie
\cite{lit04,lit01,lit02,lit03}%
\gruppe\abelsch\index{Kategorientheorie}.
Hier steht eine URL, die man in der DVI- oder PDF-Datei
auch anklicken kann:
\href{http://www.cs.tu-bs.de/}{WWW-Seiten der Informatik}
